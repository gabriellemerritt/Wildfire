
\documentclass{article}

% use Times
\usepackage{times}
% For figures
\usepackage{graphicx} % more modern
%\usepackage{epsfig} % less modern
\usepackage{subfigure} 

% For citations
\usepackage{natbib}

% For algorithms
\usepackage{algorithm}
\usepackage{algorithmic}

% As of 2011, we use the hyperref package to produce hyperlinks in the
% resulting PDF.  If this breaks your system, please commend out the
% following usepackage line and replace \usepackage{icml2014} with
% \usepackage[nohyperref]{icml2014} above.
\usepackage{hyperref}

% Packages hyperref and algorithmic misbehave sometimes.  We can fix
% this with the following command.
\newcommand{\theHalgorithm}{\arabic{algorithm}}

% Employ the following version of the ``usepackage'' statement for
% submitting the draft version of the paper for review.  This will set
% the note in the first column to ``Under review.  Do not distribute.''
\usepackage{icml2014} 


% The \icmltitle you define below is probably too long as a header.
% Therefore, a short form for the running title is supplied here:
\icmltitlerunning{Avakian, Merritt, Reeves}

\begin{document} 

\twocolumn[
\icmltitle{Project Status Report}

% It is OKAY to include author information, even for blind
% submissions: the style file will automatically remove it for you
% unless you've provided the [accepted] option to the icml2014
% package.
\icmlauthor{Lev Avakian}{email@yourdomain.edu}
\icmlauthor{Gabrielle Merritt}{email@coauthordomain.edu}
\icmlauthor{Elizabeth Reeves}{email@coauthordomain.edu}

% You may provide any keywords that you 
% find helpful for describing your paper; these are used to populate 
% the "keywords" metadata in the PDF but will not be shown in the document
\icmlkeywords{Wildfire Project, machine learning}

\vskip 0.3in
]

\begin{abstract}
Our project aims to create a learning algorithm that can predict the likelihood, location, and severity of wildfires in the state of California. On average, the state of California loses over 100 million dollars and 218,000 acres\cite{calfire11} due to wildfire damages. Currently California only employs fire prevention methods such as restricting certain kinds of fuels, controlled fires, and fire education to curb the damage of wildfires; however, using machine learning to learn fire patterns can help fire departments properly allocate resources and take targeted measures to preventing large wildfires.
\end{abstract} 

\section{Status Report}
\subsection{Problem Statement }
\subsection{Data selection} 
\subsection{Algorithm Selection}
\subsection{Steps Going Forward}

\section*{Acknowledgements} 
None

\bibliography{example_paper}
\bibliographystyle{icml2014}

\end{document} 
